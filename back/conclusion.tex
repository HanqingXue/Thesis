% !Mode:: "TeX:UTF-8" 
\begin{conclusions}
生物通路是细胞中分子间的一系列活动,导致细胞内某种产物或变化。生物通路可以导致新的分子的组装(如脂肪和蛋白质)、控制基因的表达、刺激细胞的移动等。复杂疾病往往和生物通路网络之间存在密切的关系。因此深入研究生物通路网络对于探索疾病的发病机制具有重要的意义和研究价值。生物通路网络扩展算法是重要的生物通路分析方法,生物通路网络扩展算法有助于研究生物通路和复杂疾病之间的关联。然而,传统的生物网络扩展算法存在效率低和扩展效果不佳等问题。另一方面,研究者对于通路网络可视化系统具有很大需求,而现行通路可视化系统存在着授权费用高,交互体验差等问题。

我们在本文中提出了生物通路网络的构建方法,并使用复杂网络的分析方法对生物通路网络进行分析。经过分析我们发现生物通路网络具有小世界网络的特点:图中大部分的节点不与彼此邻接,但大部分节点可以从任一其他点经少数几步就可到达。同时由最短路径分析我们得知大部分生物通络网络最短路径的平均值是2-3,说明在生物通路网络扩展算法的实际中应该着重的考虑节点的邻居、二度邻
居的重要性。同时,通路网络的聚集系数分析显示,这些网络中的聚集系数可以作为网络扩展的重要依据。

基于网络分析的基础,我们提出了一种基于深度优先策略的生物通路网络的扩展算法, 并将提出的扩展算法和有限随机游走算法、DrugNet,基于公共邻居的链接预测算法进行比较。实验表明,我们的算法与有限随机游走相比时间复杂度更低,计算速度更快。我们的算法扩展效果要比DrugNet和基于公共邻居的链接预测算法更好,兼顾了扩展效果和时间性能。

传统的生物通路网络一般是由生物学专家来绘制,数据库中展示的生物通图也是静态的不可以编辑的。当发现一个通路中有新信息时候,更新生物通路十分耗时。随着互联网技术发展和大数据时代的到来,数据可视化技术日益成熟。互联网时代研究者对于数据的快速获取和分析需求日增长。我们开发的一款基于 Cytoscape 可视化技术的生物通路网络可视化系统,该系统可以实现网络可视化、网络编辑、网络导出、网络中元素的样式修改、网络分析,同时该系统关联了基因、疾病、药物、通路、基因表达信息,系统实现了以上这些丰富信息的融合。


\end{conclusions}
