% !Mode:: "TeX:UTF-8" 
\chapter{绪论}[Example]

\section{课题背景及研究目的和意义}[Number]
生物通路是细胞中分子间的一系列活动,导致细胞内某种产物或变化。生物通路可以导致新的分子的组装(如脂肪和蛋白质)、控制基因的表达、刺激细胞的移动等\footnote{http://www.genome.gov/27530687}。复杂疾病往往和生物通路网络之间存在密切的关系。复杂疾病诸如糖尿病、癌症、心脏病,高血压等与多种致病基因、蛋白、生物通路网络是相互关联的\cite{jin2011systematic}。因此深入研究生物通路网络对于探索疾病的发病机制具有重要的意义和研究价值。随着高通量技术的发展和大量生物实验的开展,基因、蛋白、代谢等组学数据日益积累,研究人员推出了一批高质量的生物通路数据库如KEGG\cite{kanehisa2008kegg}(一种受到广泛欢迎的被生物学家广泛使用的数据库)、Reactome\cite{croft2013reactome}(一个免费的和手工标注生物通路的在线数据库);DrugBank\cite{wishart2006drugbank}(一种综合性的包含大量的药物和靶点数据以及生物通路数据的数据库)等(见表\ref{table1} )。这些数据库蕴含着大量通路、疾病、药物等信息,诸如疾病致病基因,分子间的作用信息,通路网络的拓扑结构等,对于这些数据的深入研究和挖掘对揭示复杂疾病发病机理,发现药物靶点等方面具有重要的意义。

\begin{table}[htbp]
  \centering
	\caption[table1]{常用的通路数据库及网址}
\vspace{0.5em}\wuhao
\begin{tabularx}{1.0\textwidth}{lXXl}
\toprule[1.5pt]
数据库名 & 简介 & 网址 & 参考文献\\
\midrule[1pt]
KEGG & 被生物学家广泛使用的通路数据库 & http://www.kegg.jp & \cite{kanehisa2008kegg} \\
Reactome	& 免费的和手工标注生物通路的在线数据库	& https://reactome.org & \cite{croft2013reactome} \\
WikiPathways	& 使用了维基百科概念的通路数据	& https://www.wikipathways.org & \cite{pico2008wikipathways} \\
PhosphoSitePlus	& 包含小鼠和人类的通路数据的数据库 & https://www.phosphosite.org &	\cite{hornbeck2011phosphositeplus} \\
BioCyc 	& 基因组通路数据库& https://biocyc.org/	& \cite{krummenacker2005querying} \\
PANTHER	& 基因及其产物相关的通路数据库&http://www.pantherdb.org	& \cite{mi2016panther} \\
\bottomrule[1.5pt]
\end{tabularx}
\end{table}

生物通路网络通常和多种疾病、药物、靶点等数据关联,为发现与已知的通路网络关联的药物、靶点、基因关系等,需要对已知的生物通路网络进行扩展。通路网络的常见的扩展方法是将生物通路数据映射到其他生物网络(如疾病、基因、药物网络)上,作为种子节点进行扩展(如图\ref{fig1}所示)。图\ref{fig1}中蓝色的节点是通路网络中原有的节点,红色的节点是扩展后的节点。扩展前的生物通路网络如图\ref{expansion11}所示,扩展后的生物通路网络如图\ref{expansion12}所示。各种生物网络为生物通路的扩展提供了拓扑结构信息,关联数据信息、作用强度信息等。各种生物网络的拓扑结构可以预测生物通路网络中潜在的关联,生物网络中的相互作用则可以发现的生物通路网络中未知的基因、药物、疾病等潜在的链接。将这些信息进行合理的分析和应用是进行生物通路网络扩展的基础。

 生物网络的拓扑结构信息是进行生物通路扩展的重要依据。常见的扩展方法有基于网络传播的方法和基于网络节点聚类的方法。这些方法利用了网络在结构上的全部或者部分信息去寻找生物通路节点和网络中潜在的节点之间的关联,以实现生物通路的扩展。经过扩展的生物通路包含了一些潜在的关联信息。这些潜在的信息揭示了一些复杂疾病的发病机制、某些药物的靶点等。

为了对生物通路网络进行直观的认识,便于研究人员的研究和使用。网路可视化技术常常被引入到生物网络的展示过程。生物通路网络展示可以借助于生物通路图。生物通路图是一系列化合物和反应所组成的复杂网络的一种可视化表示形式。因此,一条生物通路可表示成节点和边的集合,节点可用来表示参与化学反应的反应物和产物,例如: 蛋白质,小分子化合物,DNA,RNA 等;边可表示各种相互作用关系,例如: 反应和规则。生物通路图以直观网络图谱的方式来显示相关的生物学信息,便于生物学家认清各种生物反应的网络图谱。了解生物通路中参与反应的生物分子之间的关系,各个生物分子的功能,以及各自的反应方式,以便于生物学家对生物实验数据进行观察、记录和分析。

在生物通路图中,每个节点代表一种生物分子。图中的节点有许多相关的属性可供选择设置,例如: 形状、背景颜色、大小、边框颜色、边框层次和边框的粗细等等。在生物通路图中,用边来表示各种相互作用关系、反映类型和生物学功能等等。边可分为有向边和无向边。边也可以采用不同的形状,附加采用不同的颜色和粗细。生物通路图为研究者提供了直观的认识,作为一种重要的工具加速了生物通路的研究,方便了研究者挖掘和发现已有的通路数据中潜在的信息,在揭示疾病的发病机制,提高临床治疗效果、发现药物靶点等方面具有重要的意义。

\begin{figure}{\label{fig1}}
  \centering
  \begin{minipage}{.95\linewidth}
    \setlength{\subfigcapskip}{-1bp}
    \centering
    \begin{minipage}{\textwidth}
      \centering
      \subfigure{\label{expansion11}}\addtocounter{subfigure}{-2}
      \subfigure{\subfigure[扩展前的生物通路网络]{\includegraphics[width=0.6\textwidth]{1-a}}}
      \subfigure{\label{expansion12}}\addtocounter{subfigure}{-2}
      \subfigure{\subfigure[扩展后的生物通路网络]{\includegraphics[width=0.8\textwidth]{1-b}}}
    \end{minipage}
	\caption[fig1]{通路网络扩展实例}
  \end{minipage}
\end{figure}

\section{国内外研究现状分析}[Current research situation]
\subsection{生物通路网络分析和扩展方法的国内外研究现状}[Network analysis]
 随着生物通路数据的积累和相关数据库的日益完备,对生物通路数据分析技术和生物通路网络分析和扩展方法的需求日益增长。生物通路分析技术已经成为深入研究生物学差异基因表达和蛋白的首选,因为它不仅降低了研究成本,而且增加了对于一些现象的解释能力。通路分析技术已经被广泛的使用到对基因本体分析、生物互作网络分析(如蛋白网路)等领域。第一代的生物通路分析技术是以以ORA\cite{goeman2007analyzing}方法为代表。该方法的工作流程一般是使用一定的阈值和标准创建输入列表。然后输入一个基因的列表,使得每个列表里的基因都是生物通路的一部分,接下来对于每一个通路在输入基因列表下,进行相关的测试得到最后和输入基因最为显著相关的通路,最常见的是超几何分布、卡方检测、二项式分布检测等。然而这类的方法是具有显著的缺陷的。首先,采用不同的测试的状况下其量度不同,这就意味着除了测试基因在数量上的信息,其他的测试量之间无法进行统一的比较。其次,这种方法在考虑基因时只选择了显著的基因,而丢失其他的基因。第三,没有考虑到列表里面基因之间的关联性和通路之间的关联性。

 由于ORA技术存在的这些缺点,更为先进的通路分析技术被发展出来,其中以FCS\cite{lee2011prioritizing}为典型的代表。FCS克服了ORA缺点,并实现了在从分子测量角度寻找到生物通路上的显著基因。FCS用一个统一的分数将各种不同的统计分析方法得到的结果结合了起来。OFA 和FCS方法仅仅考虑了生物通路上基因的数量和基因表达信息来确定关键的生物通路。然而,生物通路往往和其他的基因之间存在着链接,这些基因往往和生物通路起作用。为了添加更多有用的信息,基于生物通路网络拓扑结构的方法(基于通路网络分析的方法)越来越受到重视,该方法充分利用拓扑结构信息来计算在基因层面的特性。

为了深入的研究生物通路网络,对生物通路网络结构的分析是十分重要的环节。由于生物通路网络,也是一种基于边和节点的图模型。分析传统的网络的方法,对于生物通路网络的分析过程也是十分有益的。传统的网络分析方法可以完成网络抽取、网络的中心性分析,社区探测、分类、链接预测等。这些也适应于方法在生物通路网络的分析过程中。2004年Newman\cite{newman2006modularity}等人于提出了模块度(modularity)这一概念,.这一个概念在研究者引入到复杂网络中社区发现过程中,并在成功地在社交网络分析和生物网络分析中得到了实践Tamas Nepusz\cite{nepusz2012detecting}等提出了一个基于贪心策略的子网络节点聚类算法ClusterONE。ClusterONE通过定义一个量化的度量“内聚力”,聚力是用于表征子网内部的凝集程度和子网与外部的分割程度。Wu等人利用KEGG\cite{kanehisa2008kegg}中的药物-靶点,疾病-基因关系构建了一个多重网络,使用ClusterONE和Louvain\cite{blondel2008fast}算法在这个网络进行“模块发现”,对潜在的关联关系进行了预测。Emig \cite{emig2013drug}等人提出了一种基于结合局部和全局网络信息的网络分析方法。该方法融合药物靶点、基因表达芯片数据、疾病信息构造全局网络。利用邻居得分(局部网络方法)、节点关联性(局部网络方法)、网络传播(全局网络方法)、随机游走(全局网络方法)四种方法分别对全局网络进行分析,预测了网络中潜在的链接。

基于随机游走的一类方法在通路网络扩展和通路中潜在的关联关系的预测应用方面具有重要的价值。所谓的随机游走就是在规则的点阵上进行无规律行走的模型,该模型的每个步骤,从一个位置跳转到另一相邻位置,位置变化形成的一个序列如图\ref{fig2}所示,图中红色节点表示当前时刻点所在的位置。一维的随机游走可以看成马尔科夫链,其状态空间和整数T有关系,并且状态的转移概率(从状态T 转移到状态J 的概率)由式\ref{eq:1}给出
\begin{equation}\label{eq:1}
	P_{T+1,T} = 1-P_{T,T-1}
\end{equation}

\begin{figure}
\centering
\includegraphics[width = 0.6\textwidth]{rw}
\caption[fig2]{随机游走的实例}
\end{figure}
基于随机游走概念提出的一系列方法被广泛应用在生物网络扩展和生物网络中潜在的链接预测领域,例如药物的重定位领域。本人在[]中系统的总结了基于网络的药物重定位方法,这些方法对于药物-疾病关系,疾病-通路关系的发现具有重要的意义,这些方法也能很好的迁移到生物通路网络的扩展方面。例如Chipman\cite{chipman2009predicting}等人生物网络上使用了随机游走算法,该方法捕获了生物网络的全局信息,预测出和已知通路相关的部分基因。Macropol\cite{Macropol}提出了一种自启动的随机游走算法,在蛋白质网络上检测到了相关的功能模块(蛋白质通路)。Liu\cite{liu2010link}提出了一种本地随机游走算法,该算法以随机游走算法为基础,实现了在规模的复杂网络中的关联关系的预测,以此来实现生物通路的扩展。该方法具有以较低的复杂度得到高精度的预测结果和扩展结果等特点。


国内学者也对生物通路网络的扩展进行深入的研究。王夏\cite{王夏2009大肠杆菌}等利用马尔科夫聚类的方法在对蛋白质网络进行了综合分析,借助模块分析的方法,预测了蛋白质网络中的相关功能模块,作者整合模块之间的作用关系、GO数据库注释信息、KEGG\cite{kanehisa2008kegg}数据库中的代谢通路数据,提出了一种既能扩展通路网络,又能进行致病性和细胞进程研究的方法,对于生物通路的研究具有十分重要的作用和意义。

李寅珠\cite{李寅珠2012基于蛋白质相互作用网络的代谢}等提出了利用蛋白质相互作用信息来预测扩展生物通路网络的方法。作者在使用蛋白相互作用关系构建的网络上,考虑到蛋白质网络中的一阶、二阶特性,使用聚类分析算法进行了蛋白质网络的功能划分,使得同一个生物通路网络中的蛋白均出现在同一个蛋白质功能模块,以此来发现与通路网络相关的新的蛋白。

郑伟\cite{郑伟2010一种基于随机游走模型的多标签分类算法}等人提出了一种多标签游走算法,即基于随机游走算法的多标签分类算法:首先将含有多标签的复杂网络映射成为多标签随机游走图,然后每当一个未分类(即不具有分类标签)的数据被输入时便建立一个对应的多标签随机游走图序列;最后,再对得到的图序列中每个图建立其相应随机游走模型,得到遍历每个顶点的概率分布,并将其转换为每个标签对应的概率分布,有效地解决了多标签分类的问题。该方法可以通过分类问题解决通路网络扩展问题。

谢\cite{xie2012prioritizing}等人提出了一种叫做Bi-Random Walk的方法,该方法在疾病构成网络上实现了疾病未知链接的预测,可借鉴到生物通路网络的扩展过程中。张巧生\cite{zhang2016network}提出了一种基于网络的生物通络扩展方法,该方法避免了已有的通路分析方法的局限性,实验结果表明该方法具有较强的可靠性和预测精度。

\section{生物通路网络的可视化技术国内外研究现状}[Current research situation version]

随着对生物信息研究的继续,对于通路可视化需求日益增长。在传统的情况下,生物通路网络一般是由生物学专家来绘制,数据库中展示的生物通路图也是静态的不可以编辑的。这种方式很耗时耗力,当发现一个通路中有新的信息时候,更新生物通路十分耗时。针对这些问题大量的可视化工具被开发出来,以IPA为代表的大量通路分析和可视化软件被开发出来,极大的便利了研究者的对生物通路网络图的深入研究,大大节省了研究者的时间和精力。目前主流的通路可视化平台如表\ref{table2}所示

\begin{table}[htbp]
  \centering
	\caption[table2]{当前主流的通路可视化工具}
\vspace{0.5em}\wuhao
\begin{tabularx}{1.0\textwidth}{lX}
\toprule[1.5pt]
名称 & 网址 \\
\midrule[1pt]
Ingenuity Pathway Analysis(IPA)	& www.ingenuity.com\\
GeneGo/MetaCore 	& www.genego.com\\
Pathway Studio 	& www.ariadnegenomics.com\\
GenMAPP	       & www.genmapp.com\\
WikiPathways & www. wikipathways.org\\
Pathway Painter	 & www.pathway.painter.gsa-online.de/)\\
cPath	& www.cbio.mskcc.org/cpath\\
GeneGo/MetaCore 	& www.genego.com\\
Pathway Studio 	& www.ariadnegenomics.com\\
GenMAPP	& www.genmapp.com\\
BioCyc	& www.biocyc.org\\
Pubgene	& www.pubgene.org\\
PANTHER	& www.pantherdb.org\\
WebGestalt	& www.genmapp.com\\

\bottomrule[1.5pt]
\end{tabularx}
\end{table}

当前比较主流的生物通路展示方式多基于JavaScript的可视化技术,因为互联网技术的日益发展,传统的桌面端展示方式不适合人们对信息和数据快速获取和分享的需求,而JavaScript为代表的网页端开发语言,可以实现可视化系统的线上部署访问,用户无下载软件带来时间成本,同时也省去了部分软件在不同的操作系统,不同的架构下的部署使用的繁琐步骤,因此该基于JavaScript的可视化技术越来越受到开发者用户的欢迎,常用可视化库有D3.js, cytoscape.js等。


Cytoscape\cite{}是一款受到用户喜欢的通路网络可视化软件,该软件提供了强大而丰富网络可视化和分析功能,并且实现了跨平台,该软件的研发团队为适应广大的用户的需要开发了Cytoscape.js,也是就是该软件的核心JavaScript库,该库包含了网络布局调整,网络交互、网络下载、数据下载等众多功能,用户基于此库结合自己的需要开发相关的控件、插件、布局配置等,本文第四章所开发的可视化系统也基于该库。


D3.js\cite{}是一款基于数据驱动的前端JavaScript库,该软件库提供了标准的接口封装,用户可以根据自己的需求定制开发自己所需的前端控件, 为用户个性化需求提供了极大的便利。相对于cytoscape 该软件库更加灵活,但是与之带来不便之处是使用该库进行开发工作量比较大,对于系统中具体的实现细节,需要开发者自己去处理,因此对于不熟悉前端技术的开发人员该技术具有较高的门槛。

JavaScript技术为通路可视化提供了极大的便利,同时与生物通路相关数据格式协议,也极大的方便了开发者进行可视化系统的搭建。SBGN\cite{}(系统生物学图形表示法)提供了生物化学和细胞过程可视化表征的一个标准。该标准为生物通路数据提供了同一数据格式,通路数据只需按照这个标准进行存储, 在多个平台和系统下都能进行美观的展示,免去了不同数据展示库在数据格式变化情况下引起的问题,同时基于此标准开发出来的工具可以方便更多用户和研究者使用。

\begin{table}[htbp]
  \centering
	\caption[table2]{基于SBGN标准通路网络可视化工具}
\vspace{0.5em}\wuhao
\begin{tabularx}{1.0\textwidth}{lX}
\toprule[1.5pt]
软件名称 & 网址 \\
\midrule[1pt]
Arcadia	 &  http://arcadiapathways.sourceforge.net/\\
Beacon Pathway Editor	 & https://bioinformatics.cs.vt.edu/beacon/\\
BIOCHAM	 & http://contraintes.inria.fr/biocham/\\
Biographer	 & http://biographer.biologie.hu-berlin.de/\\
BioUML	& http://www.biouml.org/\\
CellDesigner & http://www.celldesigner.org/\\
COPASI	& http://copasi.org/\\
CySBGN	 & https://www.ebi.ac.uk/saezrodriguez/cno/cysbgn/\\
Dunnart	 & http://users.monash.edu/~mwybrow/dunnart/\\
EscherConverter		& https://escher.readthedocs.io/en/latest/escherconverter.html\\
iPathways	 & http://www.ipathways.org/\\
JWS Online	& https://jjj.bio.vu.nl/\\
Mimoza	 & http://mimoza.bordeaux.inria.fr/\\
Newt Editor	& http://newteditor.org/\\
PathVisio	& http://www.pathvisio.org/plugin/sbgn-plugin/\\
PathwayLab	& http://www.innetics.com/\\
SBGN-ED	  & https://immersive-analytics.infotech.monash.edu/vanted/addons/sbgn-ed \\
SBML Layout Viewer	 & http://sysbioapps.dyndns.org/Layout/\\
SBMM assistant	 & http://www.sbmm.uma.es/SPA/\\
yEd Graph Editor	& https://www.yworks.com/products/yed\\

\bottomrule[1.5pt]
\end{tabularx}
\end{table}

国内在生物通路网络的可视化方面也有诸多进展。竺涌楠\cite{}等人设计并实现了一套基于html5 的通路展示系统;胡言石\cite{}等人实现了一种与帕金森相关的生化通路和蛋白质互作网络的可视化系统;黄益灵\cite{}等人实现了一个名为PBSK 的浏览器,该浏览器可以实现四种XML格式的生物通路数据的展示。

\section{主要研究内容}
本课题主要研究的内容是生物通路网络的扩展算法的研究和生物通路网络可视化平台的设计与实现。本课题兼顾到在算法方面的创新性和在平台实现的实用性。

生物通路是细胞中分子间的一系列活动,导致细胞内某种产物或变化。一些通路网络会引起新的分子的组装比如脂肪和蛋白质的生成。生物通络可以控制基因的打开,或者刺激细胞的移动。由于复杂疾病往往和生物通路之间具有密切的关系,与生物通路相关联的多种的基因、蛋白等对于疾病的发病机制具有重要的影响。随着高通量技术的发展,基因、蛋白、代谢等组学数据日益积累,结合这些丰富的生物数据进行生物通路的扩展是一项重要研究手段,对于发掘生物通路潜在的功能,在疾病发病机制研究、临床治疗水平的改善、药物研究等领域具有重要的作用。

目前,有关通路网络的扩展算法主要集中在生物通路网络中链接的预测和生物通路网络的聚类分析,模块发现。这些方法没有很好的利用生物通路网络的拓扑结构关系及网络的自身特点,部分算法时间复杂度过大,计算时需要的软硬件资源过多。因此,提高算法的性能,改善生物通路网络的扩展效果的迫在眉睫。就通路网络可视化技术而言,部分生物通路网路展示软件,属于商用软件的范畴,使用的资金费用过高,部分开源的软件项目,功能又过于零散,因此急需一款较为综合的通路网络可视化和分析软件。因此,本研究从生物网络的特点出发,旨在提出一种高效的快速的生物网络扩展算法,同时将开发一个生物通路网络的可视化展示系统与生物通路网络扩展算法相互结合,便于用户的使用,将算法创新与工程实践相融合。

本文各章节的内容安排如下:

第二章 生物通路网络分析:本章旨在分析生物通路网络的拓扑结构,分析重要网络参数,为第三章算法设计做铺垫,同时为第四章系统中的网络分析部分做准备。

第三章 设计与实现融合生物通路网络特点的扩展算法,在算法的时间性能、扩展质量等方面与已有算法进行对比

第四章 生物通路网络可视化系统。详细的介绍可视化系统的架构、设计理念、开发技术等。


\chapter{生物通路的构建和分析方法研究}
\section{引言}
随着生物通路数据的积累和相关数据库的日益完备,生物通路分析已成为研究的重点。传统的生物通路分析技术没有考虑到通路中基因之间的相互作用对整个通路的影响,因此其分析效果和分析能力有限。随着对复杂疾病的研究日益深入,研究者发现复杂疾病往往是和多种基因相关的,因此使用网络的方法对复杂疾病进行研究成为了当前通路研究的主流。根据美国国家图书馆的统计\footnote{website},研究生物网络的文献数目呈现出爆发式的增长趋势,对生物通络网络研究文献也在逐年上升。

生物通路网络是复杂网络中的一种,对于复杂网络的分析的方法可以迁移到对于生物通路网络分析当中。生物通路网络中的作用关系是动态变化的,因此反应在网络中,通路网络节点之间的作用会随着时间和外部条件的不同而变化,因此网络具有一种动态的特征,这种特点和复杂网络的特征是相符合的,因此复杂网络的分析方法在生物通路网络同样适用。生物通路网络的拓扑属性、 网络特性、重要网络参数等可以作为通路网络进行扩展的依据。

本章中我们讨论了生物通路网络的构建方法,基于建立的生物通路网络进行了网络拓扑结构、网络参数、中心性等分析,进而为生物通路网络扩展算法的研究奠定基础。

\section{数据的选取与分析}
生物通路网络的分析方法借鉴了一般网络的分析和扩展方法,将生物通路网络背景建立在已有的生物网络中,借助对这些网络分析,以此达到扩展通路网络的目的。因此,选择合适的通路数据库和网络结构数据库是十分重要的,在本文中,我们适用包含大量可靠的PPI关系的生物网络HumanNet\cite{}作为基础网络,同时选取KEGG\cite{}数据库作为通路数据的来源, 同时作为验证数据使用BRCA作为数据来源,以下我们将这几个数据库进行详细介绍。

KEGG\cite{}Kyoto Encyclopedia of Genes and Genomes)是一套日本于1995年制定的人类基因组计划,此为关于基因组、酶促途径以及生物化学物质的在线数据库。其途径数据库PATHWAY之中记录的是细胞之中的分子相互作用网络以及具体生物所特有的变化形式。KEGG\cite{}提供了一个分类体系,在同一条通路上的有相似的或者相同的功能蛋白质会被归为一组,在本文中KEGG中数据会作为通路扩展的种子节点。

HumanNet\cite{}是韩国延世大学和奥斯汀德克萨斯大学的联合研究成果。HumanNet\cite{}整合了21种不同物种的组学数据,根据这些生物数据与已知的人类基因之间的功能作用强弱,这些数据被赋予不同的权重,并通过改进的贝叶斯方法构建了网络结构,形成了一个概率功能网络。HumanNet\cite{}囊括了18 714个编码基因,25 421对基因间的相互作用,在HumanNet\cite{}中每个节点代表一个编码基因,每一条边代表两基因间的相互作用并被赋予了权重,该权重是一个相关的对数似然得分,表征两基因间相互作用的概率。HumanNet\cite{}为基因的优先选择提供了一个通用的方法,对于一个给定的基因,HumanNet\cite{}会提供一个该基因与其他基因关联的次序,便于挖掘疾病在基因层面的关联关系。HumanNet\cite{}也为研究者提供了用户接口(http://www.functionalnet.org/humannet)以方便研究人员进行相关的研究工作。由于该网络融合了多种类型的信息,并且具有较强的通用性,在这篇文章中我们采用HumanNet\cite{}作为基础网络构建我们的功能网络。

BRCA数据集是从癌症基因组网站TCGA(英文名全称:The Cancer Genome Atlas,网址为http://cancergenome.nih.gov/)下载的数据,该数据由590个样本组成,其中521个是乳腺癌样本,61个样本是符合正态分布的样本。该数据由于数据质量较好,被广泛的应用于癌症通路的识别,药物靶点发现,药物重定位的研究。

\section{网络的构建方法研究}
我们选择HumanNet作为基础网络,基因关联网络已经被广泛的应用于理解疾病。HumanNet是一个综合了人类基因的网络,我们在HumanNet上获取基因对之间的相互作用的强度,在HumanNet中基因间的作用强度用一个对数似然分值S表示,S的值是一个大于1 的实数,为了避免由量度引入的误差,我们采用离差标准化将基因间的强度进行归一化由式\ref{eq21}给出
\begin{equation}\label{eq21}
SN=\frac{S( g_{i} \ ,g_{j} \ ) \ -\ S_{min}}{S_{max\ } -\ S_{min}}
\end{equation}

\begin{tabularx}{\textwidth}{@{}l@{\quad}r@{———}X@{}}
式中& $\boldsymbol{SN}$ & 标准化后对数相似分数值;\\
	& $\boldsymbol{S_{max\ }}$ &HumanNet中最大的对数似然分值;\\
	& $\boldsymbol{S_{min\ }}$ &HumanNet中最小的对数似然分值;\\
	 & $\boldsymbol{ g_{i}, g_{j}}$ &HumanNet中的基因;\\
	& $\boldsymbol{S( g_{i} \ ,g_{j} \ )}$ & 标准化之前对数似然分数值;\\
\end{tabularx}\vspace{3.15bp}

\begin{figure}
\centering
\includegraphics[width = 0.8\textwidth]{pathway_cons}
\caption[fig21]{通路网络的建立过程}
\label{fig21}
\end{figure}
通路网络 是一个带权图G(V, E),其中集合 V是由基因组成的集合。E是边集,E中的每一个元素$e_{ij}=(v_{i}, v_{j})$代表基因之间的相互作用。如果与 $v_{i}$和$v_{j}$对应的基因$g_{i}$和$g_{i}$ 对应的边$< g_{i} ,g_{j}  >$存在于HumanNet中那么就把边 $< v_{i} ,v_{j}  >$加入集合E中(如图\ref{fig21}所示), $w_{ij}$ 是集合W中的一个元素,它代表 $v_{i}$和 $v_{j}$作用的强弱, 被量化如下:
\begin{equation}\label{eq22}
w_{ij} ( v_{i} ,\ v_{j}) =SN( g_{i} ,g_{j})
\end{equation}
\begin{tabularx}{\textwidth}{@{}l@{\quad}r@{———}X@{}}
式中 & $\boldsymbol{ g_{i}, g_{j}}$ &HumanNet中的基因;\\
	& $\boldsymbol{SN( g_{i} ,g_{j})}$ & HumanNet中标准化后对数相似分数值;\\
\end{tabularx}\vspace{3.15bp}
至此,通路网络被建立起来。


\section{ 通路网络的分析方法}
由于本文第三章将探究通路网络的扩展算法,因此对通路网络的分析是进行算法设计的基础。作为复杂网络的一种,生物通路网络
具有复杂网络的共性,同时也有其自身独有的特点。由于复杂网络的理论已经在社交网络分析,蛋白质功能预测、个性广告推荐等领域取得了较好效果,因此这些成熟的复杂网络分析方法也将帮助我们了解生物通路网络的特点,为实现该网络的扩展,生物通路网络的部分特点也将作为网络扩展的部分依据。

复杂网络的分析主要集中在其拓扑结构、 网络参数、 中心性等方面。网络的拓扑结构可以反映网络整体的特征,一些基于网络整体信息的方法如基于信息传播的方法\cite{}都是基于网络的拓扑结构和性质; 网络的参数分析可以反应在度分布、连通域分布、共享邻居分布、最短路径分布等方面,这些网络参数既有基于局部的特点、也有基于全局的特点的。中心性则集中在对网络中节点重要性的衡量上,传统的网络扩展方法大部分基于公共邻居、基于度\cite{}等信息,反应节点在网络中的重要性质时信息过于单一,因此对于中心性的分析将帮助研究更好的衡量和评价节点的重要性。

在本节中我们将介绍通路网络的节点度分布、平均聚类系数分布、拓扑系数、最短路径分布、共享邻居节点分布、邻域连通性分布、介数中心性、接近中心性、压力中心性等属性。同时结合具体的生物通路网络实例进行简要的分析。

\subsection{度分布}
图结构中与某节点相连接的边的数目为该节点的度,而图中各个的节点度的散布情况就为度分布。度分布是节点度信息在网络中的一个总体描述。一个无向图 ${\displaystyle G=G(V,E)}$ 是指某个节点$ {\displaystyle v_{i_{0}}}$与之相连所有的节点的数目。

\begin{equation}\label{eq22}
	d( v_{i_{0}}) =\sum\limits ^{e_{i,j} \in E\ }_{i=i_{0}} e_{i,j}
\end{equation}
度分布则是对每个非负整数 ${\displaystyle m}$, 度为m的顶点在所有的顶点的比例,可以形式化如下:

\begin{equation}\label{eq23}
{\displaystyle \forall m\in \mathbb {N} ,\,\,P:m\mapsto P(m)={\frac {\operatorname {Card} \{v_{i}\,|\,d(v_{i})=m\}}{n}},}
\end{equation}
其中: Card 为计数函数。其中一个约束条件为$ {\displaystyle \sum _{m\in \mathbb {N} }P(m)=1}$

\subsection{平均聚类系数分布}
平均聚类系数表示一个图形中节点聚集程度的系数。是一个用来衡量网络节点聚类的情况的参数。 其数学定义如下
\begin{equation}\label{eq24}
CC_{v2} =\frac{2n}{_{k( k-1)}}
\end{equation}
\begin{tabularx}{\textwidth}{@{}l@{\quad}r@{———}X@{}}
式中 & $\boldsymbol{k}$ &  表示节点v2的所有相邻的节点的个数,即节点v2的邻居;\\
	& $\boldsymbol{n}$ &   表示节点v2的所有相邻节点之间相互连接的边的个数;\\
	& $\boldsymbol{CC_{v2}}$ &	节点V2的平均聚类系数值 \\
\end{tabularx}\vspace{3.15bp}

\begin{figure}
\centering
\includegraphics[width = 0.3\textwidth]{cc}
\caption[fig24]{聚集系数计算的示例图}
\label{fig24}
\end{figure}

聚集系数表示一个图中节点的聚集程度,在实际的网络中,相对高密度的连接点关系具有某种组织结构属性,聚集系数高的连接之间存在关联性的概率大于两点随机建立的连接的概率。整个网络的聚集系数是网络中所有的节点的聚集系数的平均值。图\ref{fig24}给出了一个实例图中节点b 的聚集系数计算实例。与b 相邻的节点个数是3个因此k 的取值是3; 在与b相连的节点中其中相邻的有c和d,因此n的取值是1由计算公式可得 $CC_{b}=1/3$。

\subsection{拓扑系数}
拓扑系数反映的是网络中的节点具有共享邻居的趋势。数学形式定义如下
\begin{equation}\label{eq25}
	T_{n} =\ \frac{avg( J( n,m))}{k_{n}}
\end{equation}
\begin{tabularx}{\textwidth}{@{}l@{\quad}r@{———}X@{}}
式中 & $\boldsymbol{k_{n}}$ &  表示节点n的所有相邻的节点的个数,即节点v2的邻居;\\
	& $\boldsymbol{ J( n,m)}$ &   表示节点n和m  之间共享的邻居的个数;\\
\end{tabularx}\vspace{3.15bp}
需要说明的一点是如果n没有邻居节点,那么此时n的拓扑系数为0,拓扑系数反映了一个节点和其他节点之间共享邻居的一种趋势。以图\ref{fig25}为例计算节点b的拓扑系数:由于图 中J(b,c)= J(b,d)= J(b,e)= 2故 $T_{b}=2/3$。
\begin{figure}
\centering
\includegraphics[width = 0.3\textwidth]{toplog}
\caption[fig25]{拓扑系数计算的示例图}
\label{fig25}
\end{figure}
\subsection{最短路径分布}
由于生物通路网络被看做图模型来进行研究,因此图上的最短路径求解算法都适用于生物通路网络,常用的最短路径求解算法有Dijkstra\cite{}算法,Bellman-Ford\cite{}算法,Floyd\cite{}算法等。
最短路径分布就是按照给定的算法计算图中两两之间的最短距离,统计最短路径的分布情况,最短路径分布反映出网络宏观上结构特点如小世界性等特点\cite{Article Navigation
Computing topological parameters of biological networks}。

\subsection{邻域连通性分布}
某一个节点的连通性指的是它的邻居节点的个数,某一个节点的邻域连通性指的是该节点对应所有邻居节点的连通性数值的平均值。这里邻域连通性分布指的是对含k个邻居节点的目标节点的邻域连通性值的分布(k=0,1,2……)。如果邻域连通性分布是k的递减函数 ,那么网络高连通的节点更趋于边缘化。若为递增函数,则这些高连通的节点更趋于网络中心。
\subsection{介数中心性}
介数包括节点介数和边介数两个概念。节点介数是指网络中通过该节点所有最短路径的数量与所有路径的数量之间的比例。边介数表示通过该边的所有最短路径的数量和所有路径之间的比值。介数的现实意义是反映了节点(边)在网络中的重要程度和其在整个网络中的控制能力。节点介数中心性定义如下
\begin{equation}\label{eq26}
	C_{b}( b) =\sum\limits _{s\neq b\neq t} \sigma_{st}( b) /\ \sigma_{st}
\end{equation}

\begin{tabularx}{\textwidth}{@{}l@{\quad}r@{———}X@{}}
式中 & $\boldsymbol{s,t,b}$ &  网络中的节点;\\
	& $\boldsymbol{ \sigma_{st}}$ &   表示s到t的最短路径条数;\\
	& $\boldsymbol{\sigma_{st}( n)}$ &	表示通过n的s到t最短路径条数 \\
\end{tabularx}\vspace{3.15bp}
计算中,通常除以一个:(N -1)(N -2)/ 2,其中N表示节点n参与其他节点连接的节点总数(包括n)。
在图\ref{fig25}中 $C_{b}( b) =\sum\limits _{s\neq b\neq t} \sigma_{st}( b) /\ \sigma_{st} =  0.583$

节点的介数中心性反映了该节点对网络中其他节点的控制能力,可以有效的衡量该节点在网络中的核心重要程度。
\subsection{接近中心性}
接近中心性是衡量信息从给定节点到网络中其他可到达节点的传播速度的一个度量标准,更直观地讲,它反映的是某节点到达其他节点的难易程度。节点n的接近中心性被定义为该节点到其他所有节点的最短路径长度平均值的倒数。定义孤立节点的接近中心性值为0。

\begin{equation}\label{eq27}
	Cc( b) =1/avg( L( b,\ m))
\end{equation}

\begin{tabularx}{\textwidth}{@{}l@{\quad}r@{———}X@{}}
式中 & $\boldsymbol{L(n, m)}$ &  表示n、m节点对之间的最短路径长度;\\
\end{tabularx}\vspace{3.15bp}
在图13中Cc(b) = 1/ ( (L(b, a) + L(b, c) + L(b, d) + L(b, e)) / 4) = 4/ (1 + 1 + 1 + 2) = 4/5= 0.8

\subsection{压力中心性}
压力中心性表示通过节点n的最短路径条数。图中绘制的是随着压力s值的变化,所对应的节点n的分布情况。这里s取值划分为若干个按指数幂增长的10倍的区域,所以这里横坐标取值分别为:{0}; [1, 10); [10, 100); ... 如果通过n节点的最短路径条数很大,那么节点n具有比较大的压力。



